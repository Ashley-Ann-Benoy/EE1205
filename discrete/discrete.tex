\documentclass[journal,12pt,twocolumn]{IEEEtran}
\usepackage{cite}
\usepackage{amsmath,amssymb,amsfonts,amsthm}
\usepackage{algorithmic}
\usepackage{graphicx}
\usepackage{textcomp}
\usepackage{xcolor}
\usepackage{txfonts}
\usepackage{listings}
\usepackage{enumitem}
\usepackage{mathtools}
\usepackage{gensymb}
\usepackage{comment}
\usepackage[breaklinks=true]{hyperref}
\usepackage{tkz-euclide} 
\usepackage{textgreek}                       
\usepackage{circuitikz}
\usepackage{pgfplots}                            
\usepackage[latin1]{inputenc}                                
\usepackage{color}                                            
\usepackage{array}                                            
\usepackage{longtable}                                       
\usepackage{calc}                                             
\usepackage{multirow}                                         
\usepackage{hhline}                                           
\usepackage{ifthen}                                           
\usepackage{lscape}

\newtheorem{theorem}{Theorem}[section]
\newtheorem{problem}{Problem}
\newtheorem{proposition}{Proposition}[section]
\newtheorem{lemma}{Lemma}[section]
\newtheorem{corollary}[theorem]{Corollary}
\newtheorem{example}{Example}[section]
\newtheorem{definition}[problem]{Definition}
\newcommand{\BEQA}{\begin{eqnarray}}
\newcommand{\EEQA}{\end{eqnarray}}
\newcommand{\define}{\stackrel{\triangle}{=}}
\theoremstyle{remark}
\newtheorem{rem}{Remark}

\begin{document}

\bibliographystyle{IEEEtran}
\vspace{3cm}

\title{NCERT 11.9.2  Q7}
\author{EE23BTECH11204- Ashley Ann Benoy$^{*}$% <-this % stops a space
}
\maketitle
\newpage
\bigskip

\renewcommand{\thefigure}{\theenumi}
\renewcommand{\thetable}{\theenumi}

\bibliographystyle{IEEEtran}
\textbf{Question:
Find the sum of n terms of the A.P. whose kth term is 5k + 1. }\\


\textbf{Solution:}
\textbf{
\begin{table}[htbp]
\centering
\caption{Given Data}
\label{tab:data}
\begin{tabular}{|c|c|c|}
\hline
\textbf{Symbol} & \textbf{Value} & \textbf{Parameter} \\
\hline
\(x(0)\) & \(1 \) & First Term \\
\hline
\(x(k)\) & \(5k + 1 \) & kth Term \\
\hline
\(d\) & \(5 \) & Common Difference \\
\hline
\(S(n)\) & \(?\) & Sum of \(N\) terms \\
\hline
\end{tabular}
\end{table}
}



Given:
\[
\text{kth term of AP: } a_k = 5k + 1
\]

Sequence Representation:
The given arithmetic progression (AP) can be represented as:
\begin{align}
x(n) = (5n + 1)u(n)
\end{align}
where \( u(n) \) is the unit step function.

Z-transform:
Apply the Z-transform to \( x(n) \):
\begin{align}
X(z) = \frac{5z^{-1}}{(1 - z^{-1})^2} + \frac{1}{(1 - z^{-1})}
\quad |z|>1
\end{align}
Sum of First \( n+1 \) Terms:
Express the sum of the first \( n+1 \) terms (\( y(n) \)) in terms of \( x(n) \) using the convolution:
\begin{align}
y(n) = x(n) * u(n)
\end{align}

Applying Z transform on both sides
\begin{align}
	Y(z) &= X(z)U(z)\\
	&=\frac{1}{(1-z^{-1})^2} + \frac{5z^{-1}}{(1-z^{-1})^3}
\end{align}
Using contour integration to find inverse Z transform:
\begin{align}
	y(n) &= \frac{1}{2\pi j} \oint_C Y(z) z^{n-1} dz\\
	&= \frac{1}{2\pi j} \oint_C \left[ \frac{1}{(1-z^{-1})^2} + \frac{5z^{-1}}{(1-z^{-1})^3} \right]z^{n-1} \, dz
\end{align}
The sum of the terms of the sequence is computed using the residue theorem, expressed as $R_i$, which represents the residue of the Z-transform at $ z=1 $ for the expression $ Y(z) $.
\begin{align}
	R_i=R_1 + R_2
\end{align}
 $R_1$ and $R_2$ are residues calculated at the poles of the Z-transform.
\begin{align}
		R_1 &= \frac{1}{{(2-1)!}} \left. \frac{d (z^{n+1})}{dz} \right|_{z=1} \\
	&=(n+1)
\end{align}
\begin{align}
	R_2 &= \frac{1}{{(3-1)!}} \left. \frac{d^2(5z^{n+1})}{dz^2} \right|_{z=1} \\
	&= \frac{5}{2}(n+1)(n)
\end{align}

\begin{align}
 S(n) &= R_1 + R_2\\
    &= (n+1) + \frac{5}{2}(n+1)(n)
\end{align}
\begin{figure}[h]
  \centering
  \includegraphics[width=0.6\textwidth]{figs/newstem.png}
 
  \label{fig:Stem_Plot}
\end{figure}
\end{document}
