\documentclass[journal,12pt,twocolumn]{IEEEtran}
\usepackage{cite}
\usepackage{amsmath,amssymb,amsfonts,amsthm}
\usepackage{algorithmic}
\usepackage{graphicx}
\usepackage{textcomp}
\usepackage{xcolor}
\usepackage{txfonts}
\usepackage{listings}
\usepackage{enumitem}
\usepackage{mathtools}
\usepackage{gensymb}
\usepackage{comment}
\usepackage[breaklinks=true]{hyperref}
\usepackage{tkz-euclide} 
\usepackage{listings}
\usepackage{textgreek}                       
\usepackage{circuitikz}
\usepackage{pgfplots}                            
\usepackage[latin1]{inputenc}                                
\usepackage{color}                                            
\usepackage{array}                                            
\usepackage{longtable}                                       
\usepackage{calc}                                             
\usepackage{multirow}                                         
\usepackage{hhline}                                           
\usepackage{ifthen}                                           
\usepackage{lscape}

\newtheorem{theorem}{Theorem}[section]
\newtheorem{problem}{Problem}
\newtheorem{proposition}{Proposition}[section]
\newtheorem{lemma}{Lemma}[section]
\newtheorem{corollary}[theorem]{Corollary}
\newtheorem{example}{Example}[section]
\newtheorem{definition}[problem]{Definition}
\newcommand{\BEQA}{\begin{eqnarray}}
\newcommand{\EEQA}{\end{eqnarray}}
\newcommand{\define}{\stackrel{\triangle}{=}}
\theoremstyle{remark}
\newtheorem{rem}{Remark}

\begin{document}


\bibliographystyle{IEEEtran}
\vspace{3cm}

\title{NCERT Physics 12.7. Q20}
\author{EE23BTECH11204- Ashley Ann Benoy$^{*}$% <-this % stops a space
}
\maketitle
\newpage
\bigskip

\renewcommand{\thefigure}{\theenumi}
\renewcommand{\thetable}{\theenumi}

\bibliographystyle{IEEEtran}
\textbf{
QUESTION}
A series LCR circuit with 
L = 0.12 H,
C = 480 nF, 
R=23\textOmega\
is connected to a 230 V variable frequency supply.

(a) What is the source frequency for which current amplitude is maximum. Obtain this maximum value.

(b) What is the source frequency for which average power absorbed by the circuit is maximum. Obtain the value of this maximum power.

(c) For which frequencies of the source is the power transferred to the circuit half the power at resonant frequency? What is the current amplitude at these frequencies?

(d) What is the Q-factor of the given circuit?\

\textbf{Solution: }
Given parameters are:


\textbf{
\begin{table}[htbp]
\centering
\caption{Given Data}
\label{tab:data}
\begin{tabular}{|c|c|c|}
\hline
\textbf{Parameter} & \textbf{Symbol} & \textbf{Value} \\
\hline
Inductance & $L$ & $0.12 \, \text{H}$ \\
Capacitance & $C$ & $480 \, \text{nF}$ \\
Resistance & $R$ & $23 \, \Omega$ \\
Supply voltage & $V$ & $230 \, \text{V}$ \\
\hline
\end{tabular}
\end{table}

}

(a)
\begin{equation}
\begin{aligned}
V_o &= \sqrt{2}V = 325.2 \, \text{volts} \\
\text{At resonance, } \omega_{RL} &= \frac{1}{\omega RC} \\
\omega_R &= \frac{1}{\sqrt{LC}} = 4166.67 \, \text{rad/s} \\
\nu_R &= \frac{\omega_R}{2\pi} = 663.48 \, \text{Hz} \\
I_o &= \frac{V_o}{R} = 14.14 \, \text{A}
\end{aligned}
\end{equation}

\begin{circuitikz} 
    % ... circuit diagram ...
    \draw (0,0)
    to[sinusoidal voltage source, v=$V$] (0,2) % Voltage source
    to[R, l=$R$] (2,2) % Resistor
    to[L, l=$L$] (4,2) % Inductor
    to[C, l=$C$] (4,0); % Capacitor
    \draw (4,2) to[short, -o] (5,2) node[right] {$\text{Terminal}$}; % Terminal label
    \draw (0,0) to[short, -o] (5,0) node[right] {$\text{Terminal}$}; % Terminal label
\end{circuitikz}

(b)
\begin{equation}
P = \frac{1}{2}I_o^2R = \frac{1}{2}(14.14)^2 \times 23 \Rightarrow 2299.3 \, \text{W}
\end{equation}

(c)
\begin{equation}
\begin{aligned}
\Delta\omega &= R^2L = 23^2 \times 0.12 = 95.83 \, \text{rad/s} \\
\omega'_1 &= 4166.67 + 95.83 = 4262.3 \, \text{rad/s} \\
\omega'_2 &= 4166.67 - 95.83 = 4070.87 \, \text{rad/s} \\
I &= I_0\sqrt{2} = \frac{14.14}{1.414} = 10 \, \text{A}
\end{aligned}
\end{equation}

(d)
\begin{equation}
Q = \frac{1}{R}\sqrt{LC} = \frac{1}{23}\sqrt{0.12480 \times 10^{-9}} = 21.74
\end{equation}

\end{document}

