\documentclass[journal,12pt,twocolumn]{IEEEtran}
\usepackage{cite}
\usepackage{amsmath,amssymb,amsfonts,amsthm}
\usepackage{algorithmic}
\usepackage{graphicx}
\usepackage{textcomp}
\usepackage{xcolor}
\usepackage{txfonts}
\usepackage{listings}
\usepackage{enumitem}
\usepackage{mathtools}
\usepackage{gensymb}
\usepackage{comment}
\usepackage[breaklinks=true]{hyperref}
\usepackage{tkz-euclide}
\usepackage{textgreek}
\usepackage{circuitikz}
\usepackage{pgfplots}
\usepackage[latin1]{inputenc}
\usepackage{color}
\usepackage{array}
\usepackage{longtable}
\usepackage{calc}
\usepackage{multirow}
\usepackage{hhline}
\usepackage{ifthen}
\usepackage{lscape}

\newtheorem{theorem}{Theorem}[section]
\newtheorem{problem}{Problem}
\newtheorem{proposition}{Proposition}[section]
\newtheorem{lemma}{Lemma}[section]
\newtheorem{corollary}[theorem]{Corollary}
\newtheorem{example}{Example}[section]
\newtheorem{definition}[problem]{Definition}
\newcommand{\BEQA}{\begin{eqnarray}}
\newcommand{\EEQA}{\end{eqnarray}}
\newcommand{\define}{\stackrel{\triangle}{=}}
\theoremstyle{remark}
\newtheorem{rem}{Remark}

\begin{document}
    \bibliographystyle{IEEEtran}
    \vspace{3cm}
    
    \title{GATE 23 EE Q38}
    \author{EE23BTECH11204 - Ashley Ann Benoy$^{*}$}% <-this % stops a space
    \maketitle
    \newpage
    \bigskip
    
    \bibliographystyle{IEEEtran}
    
    \textbf{Question: }
    Consider a lead compensator of the form
    \[ K(s) = \frac{1 + \frac{s}{a}}{1 + \frac{s}{\beta a}}, \quad \beta > 1, \quad a > 0 \]
    
    The frequency at which this compensator produces maximum phase lead is \(4 \, \text{rad/s}\). At this frequency, the gain amplification provided by the controller, assuming an asymptotic Bode-magnitude plot of \(K(s)\), is \(6 \, \text{dB}\). The values of \(a\) and \(\beta\), respectively, are
    
    \[
    \text{(A)} \, 1, 16 \quad
    \text{(B)} \, 2, 4 \quad
    \text{(C)} \, 3, 5 \quad
    \text{(D)} \, 2.66, 2.25
    \]
    
    \textbf{Solution:}
    \input{tables/gate_table.tex}
   
    
    \[ K(s) = \frac{1 + \frac{s}{a}}{1 + \frac{s}{a\beta}} \]
    
    \begin{align}
    K(s) &= \frac{s + a}{a} \cdot \frac{a\beta}{s + a\beta} \\
    &= \beta \frac{s + a}{s + a\beta}
    \end{align}
    
    1. If \(G(s) = \frac{k(s+z)}{s(s+p)}\) is the transfer function of a lead compensator, then the frequency at which this compensator provides maximum phase lead is \(\omega_m = \sqrt{p \cdot z}\)\\ rad/sec.
    
    2. If \(G(s) = \frac{k(s+z)}{s(s+p)}\) has to act as a lead compensator, then \(p\) must be greater than \(z\), i.e., \(p > z\).\\
    
    3. \(a+j\omega_m=j\omega_m\)\\
     
    4. \(\frac{\omega_m}{a\beta}=0\)\\
    
    Using the above properties we have:
    
    \begin{align}
    \omega_m &= \sqrt{ a \cdot a \beta}=4 \\
    \beta &> 1
    \end{align}
    
    Using gain amplification:
    \begin{align}
    K(j\omega_m) &= \frac{1 + \frac{j\omega_m}{a}}{1 + \frac{j\omega_m}{a\beta}} \\
    &= \frac{j\omega_m}{a}
    \end{align}
    
    Using gain amplification in dB:
    \begin{align}
    20\log_{10}|K(j\omega_m)| &= 20\log_{10}\left(\frac{\omega_m}{a}\right) = 6
    \end{align}
    
    Solving for \(a\):
    \begin{align}
    \log_{10}\left(\frac{\omega_m}{a}\right) &= 0.3 \\
    \frac{\omega_m}{a} &= 10^{0.3} \\
    a &\approx \frac{\omega_m}{10^{0.3}} \\
    a &\approx \frac{4}{2} \\
    a &\approx 2
    \end{align}
    
    Since \(a \approx 2\), and \(\beta\) = 4, \\
     Therefore,the correct answer is \textbf{(B) 2, 4}.
    
\end{document}

