\documentclass{article}
\usepackage{tikz}
\usepackage{amsmath}

\begin{document}
Consider the system as shown below:
\documentclass{article}
\usepackage{tikz}
\usepackage{amsmath}

\begin{document}
Consider the system as shown below:
\begin{center}
\begin{tikzpicture}
    % Box
    \draw (0,0) rectangle (4,2);

    % Arrow and label for x(t)
    \draw[->,>=stealth] (-1,1) -- node[above] {$x(t)$} (0,1);

    % Arrow and label for y(t)
    \draw[->,>=stealth] (4,1) -- node[above] {$y(t)$} (5,1);
\end{tikzpicture}
\end{center}
The system is described by the equation
\[ y(t) = x(e^{-t}). \]\\
The system is:
\begin{itemize}
    \item[(A)] non-linear and causal.
    \item[(B)] linear and non-causal.
    \item[(C)] non-linear and non-causal.
    \item[(D)] linear and causal.
\end{itemize}
\end{document}



The system is described by the equation
\[ y(t) = x(e^{-t}). \]

The system is:
\begin{itemize}
    \item[(A)] non-linear and causal.
    \item[(B)] linear and non-causal.
    \item[(C)] non-linear and non-causal.
    \item[(D)] linear and causal.
\end{itemize}

\textbf{Solution:}\\

\textbf{Homogeneity Test:}\\
For input \(x_1(e^{-t})\), the output will be \(y_1(t)\).
\begin{align}
y_1(t) = x_1(e^{-t})
\end{align}
Multiplying both sides by a scalar quantity 'a'
\begin{align}
ay_1(t) = ax_1(e^{-t})
\end{align}
For input \(x_2(e^{-t})\), the output will be \(y_2(t)\).
\begin{align}
y_2(t) = x_2(e^{-t})
\end{align}
Multiplying both sides by a scalar quantity 'b'
\begin{align}
by_2(t) = bx_2(e^{-t})
\end{align}
Adding the above equations we get:
\begin{align}
ay_1(t) + by_2(t) = ax_1(e^{-t}) + bx_2(e^{-t})
\end{align}
Let us assume that, for input \(ax_1(e^{-t}) + bx_2(e^{-t})\), the output will be \(y_3(t)\).
\begin{align}
y_3(t) = ax_1(e^{-t}) + bx_2(e^{-t})
\end{align}
But, \(ay_1(t) + by_2(t) = ax_1(e^{-t}) + bx_2(e^{-t})\)\\
Therefore;
\begin{align}
y_3(t) = ay_1(t) + by_2(t)
\end{align}

The system satisfies homogeneity, as scaling the input scales the output.\\

\textbf{Additivity Test:}\\
From the given system;
\begin{align}
y(t) = x(e^{-t})
\end{align}
\begin{align}
y(0) = x(e^{0})
\end{align}
\begin{align}
y(1) = x(e) = x(2.71)
\end{align}
So, the present value of output depends on the future value of input, indicating non-causality.

Therefore, the correct answer is:
\textbf{(B) linear and non-causal}

\end{document}
